\documentclass[11pt, a4paper, twocolumn]{article}

\newcommand{\getTitle}{Principles of Object-Oriented Programming Languages: Language Comparison}
\newcommand{\getAuthor}{Mathijs Saey}
\newcommand{\getFullTitle}{Principles of Object-Oriented Programming Languages: \\ Language Comparison}
\newcommand{\getFullAuthor}{Mathijs Saey, 94451,  \\ mathsaey@vub.ac.be, \\ 1st Master of Science in Applied Sciences and Engineering: \\ Computer Science}

%Packages
\usepackage{color}
\usepackage{graphicx} 
\usepackage{listings}
\usepackage{longtable}

%Bibliography
\usepackage{biblatex}
\addbibresource{references.bib}

%Hyperref + pdf meta info
\usepackage[hidelinks]{hyperref}

\hypersetup{
 	pdfauthor={\getAuthor},
 	pdftitle={\getTitle},
 	pdfkeywords={VUB, Object-oriented, C++, Objective-C},
 	pdfproducer=pdfLatex,
 	pdfcreator=Sublime Text 2,
}

%Extra colors
\definecolor{dkgreen}{rgb}{0,0.6,0}
\definecolor{gray}{rgb}{0.9,0.9,0.9}
\definecolor{mauve}{rgb}{0.58,0,0.82}

% Listing settings
\lstset{language=C++,
	basicstyle=\scriptsize,
	backgroundcolor=\color{gray},
	commentstyle=\color{dkgreen},
	numberstyle=\scriptsize\color{black},
	keywordstyle=\color{blue},
	morekeywords={@interface, @end, @protocol, @implementation, alloc, init, import, id},
	numbers=left,
	xleftmargin=9pt,
	framexleftmargin=9pt,
	breaklines=true,
	numbersep=1pt,
	tabsize=2}


\begin{document}

%Options
\setlength{\parindent}{0pt}
\setlength{\parskip}{1ex}

%Title + table of contents
%Title page
\title{\getTitle}
\author{\getFullAuthor}
\date{\today}
\maketitle

%Vub Logo
\begin{figure}[!]
\centerline{
\includegraphics[scale=0.2]{files/vub_logo.jpg}}
\end{figure}
\newpage
\tableofcontents
\newpage
%Content

\section{Introduction}
This paper is written for the Principles of Object Oriented Languages course at the VUB. The goal of this paper is to compare the object oriented properties of 2 languages. 

Objective-C and C++ were chosen for this purpose, these languages are both strict supersets of C, but have a very different appraoch on object oriented features. C++ is focussed towards run-time efficieny and static type checking \cite{CPdesc}; while Objective-C tries to do things dynamically whenever this is possible \cite{OCRPG}.

Both languages were compiled with 

%Metaclasses
%Everythingisanobject?       

% Pure virtual classes vs protocols
% Echte abstracte klases zijn er niet in obj-c. C++ heeft geen protocol alternatief
% Meerder @interface van zelfde klasse mogelijk? => Categorien
% nested class => in de implementatie file zetten (zoals private category)
% friend classes emuleren via category

%forwarding voor multiple inheritance? => maakt het mogelijk om een single method te inheriten

%In Objective-C, all methods are virtual. Hence, the virtual keyword does not exist and has no equivalent.

%Bibliography
\printbibliography[heading = bibnumbered]

\lstinputlisting[label=cp:typ,language=C++,caption={Templates and static typing in C++}]{../C++/Types.cpp}
\lstinputlisting[label=oc:typ,caption={Objective-C using Dynamic and Static typing}]{../objC/Types.m}

\lstinputlisting[label=cp:acc,language=C++,caption={Access rights in C++}]{../C++/Access.cpp}
\lstinputlisting[label=oc:acc,caption={Access rights in Objective-C}]{../objC/Access.m}

\lstinputlisting[label=cp:pol,language=C++,caption={Polymorphism in C++}]{../C++/Polymorphism.cpp}
\lstinputlisting[label=oc:pol,caption={Polymorphism in Objective-C}]{../objC/Polymorphism.m}

\lstinputlisting[label=cp:inh,language=C++,caption={Inheritance in C++}]{../C++/Inheritance.cpp}
\lstinputlisting[label=cp:min,language=C++,caption={Multiple inheritance in C++}]{../C++/MultipleInheritance.cpp}
\lstinputlisting[label=oc:inh,caption={Inheritance in Objective-C}]{../objC/Inheritance.m}


\end{document}
